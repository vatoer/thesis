\documentclass[12pt]{report}

% Language and encoding
\usepackage[english]{babel}

% Font packages
\usepackage{fontspec}
\setmainfont{Times New Roman}
\setsansfont{Arial}
\setmonofont{Courier New}

% Math font setup
\usepackage{unicode-math}
\setmathfont{XITS Math}       % Or use another font like 'Latin Modern Math'

% Page margins
\usepackage{geometry}
\geometry{a4paper, left=1.5in, right=1in, top=1in, bottom=1in}
\usepackage{amsmath,graphicx, xcolor}
\usepackage{enumitem}

\usepackage{etoolbox}
\apptocmd{\biburlsetup}{\Urlmuskip=0mu plus 1mu\relax}{}{}

% \usepackage{url}
% \def\UrlBreaks{\do\/\do-}

\usepackage[hyphens]{url}
\usepackage{xurl}  % Allows line breaks in URLs
\usepackage{hyperref}
\hypersetup{
    colorlinks=true,
    linkcolor=blue,
    filecolor=magenta,      
    urlcolor=blue,
    pdftitle={Overleaf Example},
    pdfpagemode=FullScreen,
    }

% Line spacing
\usepackage{setspace}
\doublespacing  % Use \onehalfspacing for 1.5 line spacing

% For figures and tables
\usepackage{graphicx}
\usepackage{caption}
\usepackage{subcaption}
\usepackage{booktabs} % For better table lines
\usepackage{tabularx}
\usepackage{array}
\usepackage{longtable} % For long tables
\usepackage{multirow} % For multi-row cells in tables
\usepackage{float} % For placing figures and tables

% For references
\usepackage{url} % For clickable URLs in the bibliography
% \usepackage{cite}
\usepackage{csquotes}         % Add csquotes for proper quotation support
\usepackage[
    backend=biber,
    style=ieee,
    url=true,
    doi=false,
    isbn=false,
    eprint=false,
    maxcitenames=2,
    maxbibnames=6
]{biblatex}
\addbibresource{references.bib}

\usepackage{lipsum} % For dummy text


\title{A Retrieval-Augmented Generation Approach with Fine-Tuned SahabatAI for Indonesian Consular Chatbot
}
\author{Fathur Rohman}
\date{May, 2025}

\begin{document}

\sloppy
\begin{titlepage}
  \begin{center}
    \vspace*{2cm} % Adjust vertical spacing as needed

    {\LARGE \textbf{A Retrieval-Augmented Generation Approach with Fine-Tuned SahabatAI for Indonesian Consular Chatbot}} \\[1.5cm]

    \textbf{Fathur Rohman} \\[1.5cm]

    \vspace{5cm} % Additional space to move the university name down
    \textbf{President University} \\[0.5em]
    Informatics Study Program \\[0.5em]
    Master of Science in Information Technology \\[0.5em]
    \textbf{Thesis Proposal} \\[0.5em]
    \textbf{2025} \\[0.5em]
    \textbf{Thesis Supervisor:} \\[0.5em]
    \textbf{Dr. Fulan, S.T., M.T.} \\[0.5em]
    \textbf{Thesis Co-Supervisor:} \\[0.5em]    
    \textbf{Dr. Fulana, S.T., M.T.} \\[0.5em]
  \end{center}
\end{titlepage}

\begin{center}
  \textbf{Abstract}
\end{center}
\begin{spacing}{1.5} % Use 1.5 line spacing for the abstract

Providing good consular services to citizen abroad is crucial for the Indonesian government. Ministry of Foreign Affairs (Kemenlu) has launched digital platforms like "Peduli WNI", "Safe Travel", and the "SARI" chatbot.
while SARI is focused on migrant workers, it may not cover all consular services queries. 
This research aims to develop a Question Answering (QA) system using a Retrieval-Augmented Generation (RAG) approach with a fine-tuned SahabatAI model(base on Gemma2 and llama3) to answer questions related to Indonesian consular services. 
The method focuses on applying Parameter-Efficient Fine-Tuning (PEFT) to adapt SahabatAI for consular topics, implementing a Retrieval-Augmented Generation (RAG) system with SahabatAI as the response generator, and building a specialized knowledge base from official MoFA sources.
The evaluation will focus on how well the system retrieves information, the quality of generated answers (accuracy and relevance), and overall performance.
The expected outcome is a robust QA system that can provide accurate and timely information to Indonesian citizens seeking consular services.

\textbf{Keywords:} Consular Services, SahabatAI Fine-Tuning, Retrieval-Augmented Generation, Fine-Tuning (RAG), Question Answering (QA) System, Public Service Automation.
\end{spacing}


\newpage

% Table of Contents
\tableofcontents
\newpage

% List of Figures (if applicable)
\listoffigures
\newpage

% List of Tables (if applicable)
\listoftables
\newpage

% Chapter 1: Introduction
\chapter{Introduction}
\section{Background}

Consular services represent a cornerstone of a nation's support for its citizens abroad. These services, which include the issuance of passports and visas, provision of emergency assistance, facilitation of self-reporting for citizens residing overseas, and management of case reports, are vital for ensuring the safety, well-being, and legal standing of individuals in foreign territories~\cite{UU37_1999,Keppres108_2003}. The Indonesian Ministry of Foreign Affairs (MoFA) is tasked with serving a substantial global diaspora and a large number of citizens traveling internationally~\cite{KPU301_2024}, which underscores the necessity for highly
efficient, accessible, and responsive support mechanisms.

In response to these demands, the Indonesian MoFA has proactively embraced digital transformation to enhance its consular service delivery~\cite{Tempo_AI_Kemlu}. This commitment is evident in its existing digital ecosystem:
% maksimum width 15.65 cm untuk content
\begin{table}[htbp]
    \centering
    \caption{Overview of Consular Service Platforms}
    \resizebox{\textwidth}{!}{
        \begin{tabular}{|p{2cm}|p{2cm}|p{5cm}|p{2cm}|p{2.5cm}|}
        \hline
        \textbf{Platform Name} & \textbf{Type} & \textbf{Key Features} & \textbf{Target Users} & \textbf{Reported AI/Technology Used} \\ 
        \hline
        Portal Peduli WNI & Web-portal & Lapor Diri (Self-Reporting), Pelayanan Kekonsuleran (Consular Services), and Pengaduan Kasus (Case Reporting) & Indonesian citizens abroad & Web-based platform \\ 
        \hline
        Safe Travel & Mobile App & Trip registration, country-specific information, notifications, emergency assistance (location sharing, video recording) & Indonesian citizens traveling abroad for short trips & Mobile application \\ 
        \hline
        SARI (Sahabat Artifisial Migran Indonesia) & Chatbot & designed capacity for empathetic responses & Indonesian female migrant workers abroad & AI-powered chatbot, NLP, integrated with Safe Travel \\ 
        \hline
    \end{tabular}
    }
\end{table}

\begin{itemize}
    \item \textbf{Peduli WNI:} This web-based platform serves as a central hub for Indonesian citizens abroad, offering critical features such as Lapor Diri (Self-Reporting), Pelayanan Kekonsuleran (Consular Services), and Pengaduan Kasus (Case Reporting). The portal has significantly streamlined processes that previously necessitated physical visits to Indonesian embassies or consulates, allowing services to be accessed online with internet connectivity~\cite{PeduliWNI}.
    \item \textbf{Safe Travel:} A mobile application designed for Indonesian citizens undertaking short trips abroad, although it can also be utilized by expatriates. It provides practical country-specific information (e.g., time differences, security conditions, local laws and customs, immigration requirements, health services at Indonesian missions), travel registration, notifications (appeals, advice, warnings), and crucial emergency assistance features. In critical situations, users can send their location, record video, and contact the nearest Indonesian mission.
    \item \textbf{SARI Chatbot:} This AI-powered chatbot represents a significant step towards leveraging advanced technology for citizen protection. Developed in collaboration with UN Women, SARI is specifically tailored to assist and protect Indonesian female migrant workers from potential violence and exploitation. Integrated within the Safe Travel application, SARI aims to deliver accessible, unbiased, and non-discriminatory information. A key feature is its designed capacity for empathetic responses. The launch of SARI underscores MoFA's commitment to "digital empathy" and delivering excellent service and protection, particularly for vulnerable groups~\cite{kemlu2025chatbot}.
\end{itemize}

The global landscape of public servce delivery is increasingly being reshaped by andvancements in artificial intelligence (AI), particularly Large Language Models (LLMs) and Retrieval-Augmented Generation (RAG) techniques. These technologies offer transformative potential including 24/7 citizen assistance, the capacity to handle complex and nuanced queries, multilingual support, and the ability to personalize interactions, thereby enhancing the efficiency and effectiveness of government services.

A recent development in Indonesia's AI journey is SahabatAI, a large language model (LLM) fine-tuned for Indonesian language tasks~\cite{gemma2_sahabat_ai_v1_instruct}. SahabatAI is collaborative initiative by Indosat Ooredo Hutchison and GoTo Group based on the Gemma2~\cite{gemma2_sahabat_ai_v1_instruct} and llama3 architecture~\cite{llama3_sahabat_ai_v1_instruct}, which has been trained on a diverse range of Indonesian text data comprising over 640,000 instruction-completion pairs, covering Bahasa Indonesia, Javanese, and Sundanese, with plans to include other regional languages like Batak and Balinese. The core objectives of SahabatAI include promoting linguistic diversity, fostering AI sovereignty for Indonesia, and enabling seamless business-to-government (B2G) and business-to-business (B2B) interactions to significantly enhance the quality and accessibility of government services. Potential use cases for SahabatAI in the public sector include simplifying applications for the national identity card (KTP), demystifying taxation processes, and streamlining procuderes for official document changes related to live events such as marriage ore relocation.

The newly existance of SahabatAI as open-source LLM, extensively trained on Indonesian text data, presents relevant technological foundation for this thesis. Developing an LLM from scratch is a complex and resource-intensive endeavor, requiring substantial computational power and extensive datasets, far exceeding the scope of a Master's thesis. By leveraging SahabatAI, this research can focus on fine-tuning the model for specific tasks, such as consular services, while also exploring the integration of RAG techniques to enhance the system's performance and responsiveness.

\section{State of the Art}

Large Language Models (LLMs), primarily based on the Transformer architecture, have revolutionized natural language processing (NLP) and artificial intelligence (AI)~\cite{NIPS2017_3f5ee243}. it has demonstrated capabilities in various tasks, including text generation, translation, summarization, and question answering. This makes them suitable for complex Question Answering (QA) tasks, where they can interpret user queries and generate relevant, coherent responses based on their vast pre-trained knowledge or context provided at inference time~\cite{jm3}. 

Retrieval-Augmented Generation (RAG) is a techniques developed to enhance the capabilities of LLMs, particularly in knowledge-intensive tasks~\cite{NEURIPS2020_6b493230}. RAG architectures connect LLMs with external, often dynamic, knowledge sources, allowing them to retrieve relevant information from a database or document corpus before generating a response. This approach addresses the limitations of LLMs, such as their fixed knowledge base and potential inaccuracies in generated content~\cite{gupta2024comprehensivesurveyretrievalaugmentedgeneration}. By integrating retrieval mechanisms, RAG systems can provide more accurate and contextually relevant answers, especially in domains where up-to-date information is crucial.
By retrieving relevant information from these sources and providing it as context to the LLM during answer generation, RAG mitigates common LLM limitations, such as hallucinations and outdated knowledge. This is particularly important in dynamic fields like consular services, where information can change over time and needs to adhere to current laws and regulations.

The application of AI in consular services is growing trend globally~\cite{govnet2025}. Governments are increasingly exploring AI-powered solutions, including chatbots for handlung frequently asked questions, assisting with visa applications, and providing real-time information to citizens. For instance, Singapore's GovTech Agency has deployed AI chatbots accross various government departments, leading significant reduce in call center workloads and faster response times for citizen inquiries~\cite{govsgvica2025}.
Similarly, the U.S. Department of State has outlined  plans to use AI for various consular function, including passport photo quality assessment, analysis of customer feedback, AI-driven translation services, and enhanced search and chatbot systems for its Travel.State.Gov website~\cite{usdosai2025}.
These examples highlight a global shift towards leveraging AI to make consular services more efficient, accessible, and responsive to citizen needs. Other relevant works include general AI Principles~\cite{molaee2025}, and AI applications in diplomacy~\cite{mostafaei2025}.

\section{Gap Analysis}

Despite the Indonesian MoFA's commendable efforts in digital transformation like the development of the "Peduli WNI" platform, "Safe Travel" application, and SARI chatbot, several gaps and opportunities remain for enhancing consular services delivery through AI. 

While the existing platforms provide valuable services, they may not comprehensively address all consular queries, particularly those related to specific legal or procedural matters. The SARI chatbot, though an innovative AI application, is specifically designed to support Indonesian female migrant workers, focusing on protection against violence and exploitation. This leaves a gap in addressing queries related to other consular services, such as passport renewal procedures, visa regulations for various countries, assistance for lost documents, and general emergency guidance without a dedicated, advanced AI-powered conversational interface. These general queries can be complex, nuanced, and often require synthesizing information from multiple official sources. Current systems may not be equipped to handle such multi-turn conversational interactions or provide comprehensive answers that require understanding implicit user needs. "Peduli WNI" while providing valuable information and transactional services, may not be fully optimized for complex queries. Futhermore, user feedback for existing applications like the "Safe Travel" app has indicated occasional technical issues, such as server connectivity issues and application crashes, suggesting room for improvements in reliability and user experience. Additionally, the current systems may not fully leverage the potential of advanced AI technologies, such as Retrieval-Augmented Generation (RAG) and fine-tuning techniques, to enhance their capabilities.

The advent of SahabatAI, an open-source LLM fine-tuned for Indonesian language tasks, with a strong foundation in Bahasa Indonesian and local dialects like Javanese and Sundanese, presents a unique opportunity to address these gaps. Appliying state-of-the-art RAG techniques in conjunction with SahabatAI model can potentially deliver more accurate, contextually relevant, and up-to-date responses to consular queries than could be achieved with generic LLMs or simple rule-based chatbot technology. The spesific combination of localized LLM like SahabatAI with advanced RAG tailored for Indonesian consular domain remains an underexplored area of research and application.

Moreover, general-purpose LLMs or even generic RAG systems often necessitate significant adaptation to perform optimally in specialized domain such as consular services, This domain is characterized by its unique terminologies, intricate regulations, evolving policies, and diverse user needs context~\cite{karzhevdatacamp2025}. Therefore, fine-tuning SahabatAI to better understand and generate text specific to consular affairs, coupled with the meticulous curation of a dedicated comprehensive consular knowledge base, is crucial for developing an effective and reliable AI assistant.

The existing MoFA digital tools—Peduli WNI (web-based), Safe Travel (mobile app), and SARI (chatbot integrated within Safe Travel)—while individually valuable, operate with some degree of separation in terms of user interface and scope. A sophisticated RAG-based chatbot, as proposed in this research, could serve as a more unified and intelligent front-end. Such a system could potentially integrate information from, or direct users to, these existing platforms, thereby providing a more seamless and comprehensive user experience for a wider range of consular questions. The knowledge base for the RAG system would ideally be constructed by consolidating information from these diverse official MoFA sources, creating a centralized and reliable information backbone for the AI. This approach could improve the discoverability and accessibility of consular information that might currently be distributed across different platforms or document formats.

\section{Research Questions}

The research questions guiding this study are as follows:
\begin{enumerate}
    \item \textbf{RQ1:} How can a RAG system using SahabatAI be designed to accurately and reliably handle various Indonesian consular queries, such as passport renewal, lost documents, and visa regulations?
    \item \textbf{RQ2:} Which PEFT strategies (e.g., QLoRA, instruction tuning) are most effective for adapting SahabatAI to the specific language and information requirements of Indonesian consular queries in a RAG framework?
    \item \textbf{RQ3:} What are the essential components and best practices for developing a high-quality, domain-specific knowledge base and QA dataset for Indonesian consular services to effectively train and evaluate a RAG system?
    \item \textbf{RQ4:} How does the fine-tuned SahabatAI-RAG system perform compared to baseline models (e.g., zero-shot SahabatAI, naive RAG) and existing MoFA chatbot solutions in terms of accuracy, relevance, coherence, and user perceived helpfulness?
    \item \textbf{RQ5:} What practical challenges and ethical considerations arise in deploying an AI-powered consular service assistant, particularly regarding data privacy, bias mitigation, and equitable information access, and how can these be addressed within a six-month Master's thesis?
\end{enumerate}

\section{Research Objectives}

To answer the research questions, this study aims to achieve the following objectives:

\begin{enumerate}
    \item \textbf{RO1:} To design and implement a RAG-based chatbot using SahabatAI to provide accurate and reliable responses to Indonesian consular queries, based on a curated knowledge base of official consular information.
    \item \textbf{RO2:} To implement and evaluate PEFT techniques (e.g., QLoRA) for fine-tuning SahabatAI using a custom Indonesian consular QA dataset, with the goal of improving its domain-specific accuracy and response quality.
    \item \textbf{RO3:} To curate a comprehensive knowledge base from official MoFA resources and develop a representative QA dataset that covers common Indonesian consular service inquiries for training and evaluating the RAG system..
    \item \textbf{RO4:} To assess the performance of the fine-tuned SahabatAI-RAG system against baseline models and MoFA chatbot solutions using automated metrics (e.g., BLEU, ROUGE) and human evaluation to measure accuracy, relevance, coherence, and user perceived helpfulness.
    \item \textbf{RO5:} To identify key ethical challenges, such as data privacy and bias mitigation, in deploying the proposed AI-powered consular assistant and to provide actionable recommendations for addressing these challenges within the scope of a six-month Master's thesis.
\end{enumerate}

\section{Limitations}

This research, while ambitious, will be conducted within the constraints of a six-month Master's thesis timeline. This necessitates a focused scope, particularly in the following areas:

\begin{itemize}
    \item \textbf{Scope of Consular Services:} The primary focus is on Indonesian consular services, including passport renewal, lost documents, and visa regulations.
    \item \textbf{knowledge Base:} The knowledge base will be constructed from publicly available official MoFA resources, with a focus on accuracy and relevance.
    \item \textbf{Language Focus:} While SahabatAI has demonstrated some training in Javanese and Sundanese, the primary focus will be on Bahasa Indonesia.
    \item \textbf{Fine-Tuning Techniques:} Given the typical constraints of computational resources and time, the research will primarily explore QLoRA and instruction tuning as PEFT strategies for SahabatAI~\cite{gemma_qlora_2025}. Full model retraining or extensive hyperparameter tuning may not be feasible within the timeframe. Recent studies suggest that even datasets around 1000 samples can be effective for PEFT~\cite{ratnakar2025qapairsassessingparameterefficient}.
    \item \textbf{Dataset Scale:} The QA dataset will be curated to cover a representative range of consular queries, but it may not cover every possible question or scenario, aiming for several hundred to a thousand samples, depending on the complexity of the questions and the available resources.
    \item \textbf{Evaluation Metrics:} The evaluation will focus on automated metrics (e.g., BLEU, ROUGE) and human evaluation for accuracy, relevance, coherence, and user perceived helpfulness. However, human evaluation may be limited to a smaller sample size.
    \item \textbf{Deployment and User Testing:} While the research will include a discussion of deployment considerations, actual deployment and extensive user testing may not be feasible within the six-month timeframe. The focus will be on developing a prototype that can be tested in a controlled environment. it will not be a production-ready system.
    \item \textbf{SahabatAI Model Version:} The research will utilize the version of SahabatAI(e.g., Gemma2 9B CPT SahabatAI v1 Instruct) available at the time of the study.
    \item \textbf{Safety and Ethical Considerations:} While the research will address ethical considerations, the implementation of safety measures and bias mitigation strategies may be limited to theoretical discussions and initial implementations, rather than comprehensive solutions.
\end{itemize}

\section{Hypothesis}

The following hypotheses will guide the empirical investigation:
\begin{enumerate}
    \item \textbf{H1:} A Retrieval-Augmented Generation (RAG) system incorporating the SahabatAI LLM, fine-tuned with a domain-specific Indonesian consular QA dataset using Parameter-Efficient Fine-Tuning (PEFT), will demonstrate significantly higher factual accuracy and contextual relevance in answering consular queries compared to the baseline SahabatAI model without RAG or fine-tuning.
    \item \textbf{H2:} The fine-tuned SahabatAI model will demonstrate improved accuracy and relevance in answering Indonesian consular queries compared to the original SahabatAI model.
\end{enumerate}

% Chapter 2: Literature Review
\chapter{Literature Review}

This chapter provides a comprehensive review of the theoretical foundations and existing work relevant to the proposed research. It delves into Large Language Models (LLMs) for Question Answering (QA), Retrieval-Augmented Generation (RAG) techniques, the specifics of SahabatAI and other LLMs, the application of AI in governmental and consular services, and methodologies for evaluating such systems.

\section{Large Language Models (LLMs) for Question Answering (QA)}

LLMs have revolutionized the field of Natural Language Processing (NLP), demonstrating unparalleled capabilities in understanding and generating human-like text. Their success is largely attributable to the \textbf{Transformer} architecture, first introduced by Vaswani et al. (2017)~\cite{NIPS2017_3f5ee243}. This architecture's core innovation, the \textbf{attention mechanism} (specifically self-attention and multi-head attention), allows models to weigh the importance of different words in an input sequence and capture long-range dependencies and complex contextual relationships, which are crucial for nuanced question answering.

The development of LLMs typically follows a two-stage paradigm:
\begin{enumerate}
    \item \textbf{Pre-training:} In this phase, LLMs are trained on vast and diverse text corpora, often sourced from the internet (e.g., Common Crawl, Wikipedia) and large book collections. The training objectives vary depending on the model architecture. Encoder-based models like BERT often use Masked Language Modeling (MLM), where the model predicts masked (hidden) words in a sentence. Decoder-based models like GPT family~\cite{radford2018improving,radford2019language,brown2020languagemodelsfewshotlearners} and Gemma ~\cite{gemmateam2024gemmaopenmodelsbased} utilize Causal Language Modeling (CLM), where the model predicts the next word in a sequence. This extensive pre-training phase endows LLMs with broad linguistic understanding, grammatical proficiency, and a significant amount of factual knowledge embedded within their parameters. 
    \item \textbf{Fine-tuning:} After pre-training, LLMs are adapted to specific downstream tasks (e.g., question answering, summarization, translation) or specialized domains using smaller, curated datasets~\cite{weng2024navigatinglandscapelargelanguage}.
\end{enumerate}

Several state-of-the-art LLMs have emerged from leading research institutions and companies, each with unique architectures and training methodologies. Notable examples include OpenAI's GPT series~\cite{openai2024gpt4technicalreport}, Metas's LLaMA series~\cite{touvron2023llama2openfoundation}, Google's Gemini~\cite{geminiteam2025geminifamilyhighlycapable} and PaLm families~\cite{chowdhery2022palmscalinglanguagemodeling}, and Anthropic's Claude models~\cite{anthropic2025claude37}. These models exhibit remarkable performance across a wide range of NLP tasks, including question answering, text generation, and summarization.

Within the Indonesian context, the SahabatAI model, developed by Indosat Ooredo Hutchison and GoTo Group, represents a significant advancement in LLM technology. 

\begin{itemize}
  \item \textbf{Architecture:} SahabatAI is based on Google's Gemma2 architecture 15, with the specific publicly available instruct-tuned model being gemma2-9b-cpt-sahabatai-v1-instruct~\cite{gemma2_sahabat_ai_v1_instruct}. Gemma models are decoder-only transformers. SahabatAI has a context length of 8192 tokens, although some evaluations have used a capped context of 4096 tokens due to inference platform limitations.
  \item \textbf{Development and Collaboration:} This LLM is collaborative effort, co-initiated by Indosat Ooredo Hutchison and GoTo Group, and developed in partnership with AI Singapore. The development leveraged NVIDIA's NeMo Framework and NIM microservices for model training and deployment~\cite{sahabatAI2024news}.
  \item \textbf{Training Data:} The model has been trained on a diverse range of Indonesian text data, including over 640,000 instruction-completion pairs covering Bahasa Indonesia, Javanese, and Sundanese. There are plans to include other regional languages like Batak and Balinese in future iterations.
  \item \textbf{Training Data \& Language Capabilities:} SahabatAI was trained with a strong emphasis on Bahasa Indonesia, using 448,000 instruction-completion pairs. The dataset also includes 96,000 Javanese and 98,000 Sundanese pairs to cover regional dialects, along with 129,000 English pairs for multilingual support. The training data combined synthetic instructions and curated public data reviewed by native speakers to maintain quality and cultural relevance. The goal is to develop models that effectively grasp local contexts and cultural nuances.
  \item \textbf{Performance \& Benchmark:} SahabatAI has demonstrated strong performance, reportedly outperforming model like Lllab-3.1-8B and sea-lionv3-9B on the SEA HELM(BHASA) evaluatioan benchmark, it has been evaluated on variety of tasks within SEA HELM(including QA, Sentiment Analysis, Toxicity Detection, Translation, Summarization, Causal Reasoning, and Natural Language Inference) and also on the IndoMMLU benchmark, which covers examination questions accross various subjects and educational levels in Indonesia. 
  \item \textbf{Availability:} SahabatAI is an open-source model, with models accessible through Hugging Face~\cite{sahabat_ai}. This open-source approach encourages community collaboration and further research in the field of Indonesian NLP. it relased under the Gemma Community License.
  \item \textbf{Limitations:} Like many LLMs, SahabatAI is susceptible to common issues such as "hallucinations" (generating incorrect or nonsensical information). A critical point for this research is that the publicly released SahabatAI model have not undegone spesific safety alignment. Developer and users are explicitly advised to conduct therir own safety fine-tuning and implement appropriate safety measuress. This is particularly important for applications in sensitive domains like consular services, where accuracy and reliability are paramount. 
\end{itemize}

To provide a broader regional context, several other Southeast Asian LLMs have been deployed, reflecting the increasing focus on local language. These include SEA-LION(AI Singapore)~\cite{ng2025sealionsoutheastasianlanguages}, SeaLLM(Alibaba)~\cite{zhang2024seallms3openfoundation}, and Sailor(SEA AI Lab \& Singapore University of Technology and Design)~\cite{dou2024sailoropenlanguagemodels}. SEA-LION, for instance, was trained on 11 regional languages, including Indonesian and Javanese, while Sailor supports Indonesian among others, These initiatives underscore the importancen of linguistic diversity and local adaptation in LLM landscapes.

The characteristic of SahabatAI—--its open-source nature, focus on Indonesian language and dialects, and its potential for fine-tuning---make it a suitable candidate for this research. However, its documented limitations, particularly the potential for hallucinations and the lack pre-existing safety alignment, are critical factors that must be proactively addressed within the proposed research methodology. This will involve leveraging RAG ground responses in factual consular data and incorporating safety guardrails in the system design.


\section{Retrieval-Augmented Generation (RAG)}

RetrievalAugmented Generation (TAG) has emerged as powerful paradigm for enhancing the capabilities of LLMs, particularly in knowledge-intensive and fact-sensitive applications. The core concept pf RAG, as introduced by Lewis et al. (2020)~\cite{NEURIPS2020_6b493230}, is to combine the strengths of retrieval-based and generation-based approaches to improve the accuracy and relevance of generated responses. In a typical RAG architecture, a retriever component is used to identify and retrieve relevant documents or passages from an external knowledge base or corpus, which are then provided as context to a generator model (often an LLM) during the response generation process.

The typical RAG pipeline operates as follows~\cite{nvidiaRAG, weka2024rag}:

\begin{enumerate}
    \item \textbf{Query Encoding:} The user's input querys is transformed into dense ventor representaion (embedding) using text embedding model.
    \item \textbf{Document Retrieval:} This query embedding is used to search a pre-indexed collection of documents (the knowledge base, often stored in a vector database). The retriever identifies and fetches the most relevant document chunks based on semantic similarity (e.g, cosine similarity between query and document embeddings).
    \item \textbf{Context Augmentation:} The retrieved document chunks are then concatenated with the original user query to form augmented prompt.
    \item \textbf{Answer Generation:} This augmented prompt is then fed to the LLM, which generates a response grounded in both the user's query and the provided contextual information.
\end{enumerate}

\begin{figure}[!htbp]
    \centering
    \includegraphics[width=0.3\textwidth]{images/1/rag-pipeline.png}
    \caption{The typical RAG pipeline.}
    \label{fig:rag-pipeline}
\end{figure}

The \textbf{benefits of RAG} anre manifold. it significantly reduces the likelihood of "hallucinations" by anchoring responses to factual data retrieved from the external knowledge base~\cite{xu2025simragselfimprovingretrievalaugmentedgeneration}.
RAG systems can access up-to-date information without the need for frequent and costly retraining of the entire LLM; the knowledgebase can be updated independently. This also enables domain specialization by simply providing a domain-spesific knowledge base. Furthermore, RAG can facilitate source attribution, allowing users to verify the information presented in the LLM's response by referring to the source documents.

RAG implementation can range from simple (Naive RAG) to more complex (Advanced RAG) techniques~\cite{gao2024retrievalaugmentedgenerationlargelanguage}:

\begin{itemize}
    \item \textbf{Naive RAG:} The most basic RAG approach, which simply retrieves relevant documents based on query similarity from the knowledge base and uses them as context for the LLM's response generation.
    \item \textbf{Advanced RAG:} This approach incorporates additional techniques at various stages of the RAG pipeline to improve performance:
    \begin{itemize}
      \item \textbf{Pre-Retrieval Strategies:} Techniques like query expansion(e.g., adding synonyms or related terms), query transformation(e.g., rephrasing the query for clarity), or generating multiple sub-queries from a complex query to retrieve a richer set of documents.
      \item \textbf{Retrieval Strategies:} Moving beyond simple dense vector retrieval to include hybrid search (combiningn keyword-based search like BM25 with semantic search), optimizing the choise and fine-tuning of embedding models, or employing graph-based retrieval mechanisms (e.g., Graph RAG, KG-RAG, where knowledge graphs are used to enhance retrieval).
      \item \textbf{Post-Retrieval/Re-rangking:} After an initial set of documents is retrieved (e.g., top-N candidates), a re-rangking model is used to to re-order these documents based on a more fine-grained assessment of their relevance to the query~\cite{Ma_2023}. Cross-encoder models, which joinly process the query and each candidate document, are often effective for this but are computationally expensive than bi-encoder retrievers.
      \item \textbf{Iterative/Multi-hop Retrieval:} For complex questions that require synthesizing information from multiple sources or performing multi-step reasoning, iterative retrieval techniques can be employed. This might involve decomposing the main question into sub-questions, retrieving evidence for each, and then synthesizing an answer~\cite{jin2025searchr1trainingllmsreason}. The Collab-RAG framework, for example, proposes using smaller language model (SLM) to decompose complex queries, with a larger LLM acting as the reader/synthesizer.
      \item \textbf{Fine-tuning RAG Components:} This involves training the retriever (embedding model) and/or the generator LLM spesifically for the RAG task and target domain. This can improve the alignment between the retriever and generator and enhance the generator's ability to utilize retrieved context effectively~\cite{xu2025simragselfimprovingretrievalaugmentedgeneration}.
    \end{itemize}
\end{itemize}

\textbf{Embedding models} play a crucial role in the RAG pipeline, as they are responsible for converting text (queries and documents) into dense vector representations. State-of-the-art embedding models include Google's Gemini Embedding (models like text-embedding-004 and the experimental gemini-embedding-exp-03-07), which has shown top performance on the MTEB Multilingual benchmark~\cite{kilpatrick2025gemini}.

Other strong open-source multilingual models like multilingual-e5-large-instruct are also widely used. The performance of these models is often evaluated on benchmarks like MTEB (Massive Text Embedding Benchmark)~\cite{muennighoff2023mtebmassivetextembedding} and its multilingual extension, MMTEB~\cite{enevoldsen2025mmtebmassivemultilingualtext}, which cover various tasks and
languages. For Indonesian, specific resources like the Indonesian Sentence Embeddings project and its associated benchmarks (e.g., SemRel2024, Indonesian subsets of MIRACL and TyDiQA) provide valuable evaluation points~\cite{wongso2024indonesian}

The generated embeddings are typically stored and queried using \textbf{vector databases}. Theses databases are optimized for efficient similarity search in high-dimensional spaces, employing Approximate Nearest Neighbor (ANN) algorithms  like HNSW (Hierarchical Navigable Small World)~\cite{malkov2018efficientrobustapproximatenearest} or IVF (Inverted File Index)~\cite{chirkin2024ivfpq}. Key features to consider when choosing a vector database include scalability, query latency, support for metadata filtering (allowing hybrid search), and ease of integration. Popular open-source options suitable for academic research include FAISS, Milvus, and Qdrant~\cite{payong2024choosing}.

Despite its advantages, RAG systems face several challenges:

\begin{itemize}
    \item \textbf{Retrieval Quality:} The "garbage in, garbage out" principle applies; if the retriever fails to fetch relevant documents or retrieves low-quality information, the generator's output will likely be inaccurate or irrelevant. A common issue is the "lost in the middle" problem, where LLMs tend to ignore relevant information if it's buried within a long contextn of retrieved documents~\cite{Zhang2025}.
    \item \textbf{Generation Quality:} Even with relevant retrieved documents, the LLM might still hallucinate, fail to synthesize information coherently from multiple documents, or produce responses that are not faithful to the provided sources~\cite{10.1145/3644815.3644945,zhou2024trustworthinessretrievalaugmentedgenerationsystems}.
    \item \textbf{Context Window Limitations:} LLMs have a finite input context windows. Effectively summarizing and presenting a large amount of retrieved information to LLM without exceeding this limit or losign crucial details is a challenge. 
    \item \textbf{Evaluation Complexity:} Evaluating a multi-stage RAG pipeline ins inherently complex, as it requires assessing the the performance of both the retrieval abd generation components, as well as their interaction~\cite{brehme2025llmstrustedevaluatingrag}.
    \item \textbf{Safety and Bias:} RAG systems can inadvertently propagate biases present in the retrieved documents. There's also a risk that even if the retrieved documents are safe and factual, a non-safety-aligned LLM might still misinterprete or "twist" this information to generate unsafe or misleading outputs~\cite{an2025ragllmssafersafety}.
\end{itemize}

Given the nature of Indonesian consular information—which often involves legal nuances, specific procedural details, and varying citizen situations—a naive RAG approach may prove insufficient. The complexity and criticality of providing accurate consular advice necessitate the exploration and implementation of advanced RAG techniques. Specifically, effective re-ranking of retrieved documents to ensure high relevance, and potentially iterative retrieval strategies for handling complex, multi-faceted queries, will be crucial for building a robust and reliable system.
Furthermore, the inherent limitations of the SahabatAI model, such as its potential for
hallucination, can be better mitigated by providing it with higher quality, more
precisely retrieved context that advanced RAG components can offer.

\section{Fine-tuning LLMs for Domain-Specific QA and RAG}

Fine-tuning pre-trained LLMs is a critical step in adapting then to spesific domains or task, such as the nuanced requirements of consular questions answering. This process adjusts the model's paramaters using a smaller, domain-specific dataset, enabling it to learn the particular vocabulary, style, and knowledge patterns relevant to the target application.

A distincion is made between \textbf{Full Fine-Tuning (FFT)} and \textbf{Parameter-Efficient Fine-Tuning (PEFT)}:

\begin{itemize}
    \item \textbf{Full Fine-Tuning (FFT):} This approach involves updating all parameters of the pre-trained model during the fine-tuning process. While effective, it is computationally expensive, requires significant memory resources, making it less practical for large models or limited hardware environments. it also requires large domain-specific datasets to avoid overfitting. A notable drawback is the risk of "catastrophic forgetting," where the model loses some of its general capabilities learned during pre-training.
    \item \textbf{Parameter-Efficient Fine-Tuning (PEFT):} PEFT methods aim to overcome the limitations of FFT by only updating a small subset of parameters or introducing additional lightweight modules while keeping the majority of the model's parameters frozen. This approach significantly reduces the computational burden and memory requirements, making it feasible to fine-tune large models on smaller datasets. Popular PEFT techniques include:
    \begin{itemize}
      \item \textbf{LoRA (Low-Rank Adaptation) and QLoRA (Quantized LoRA):} LoRA introduces small, trainable low-rank matrices into the layers of the transformer model, effectively learning task-specific adaptations altering the original weights~\cite{hu2021loralowrankadaptationlarge}. QLoRA extends this by quantizing the model weights (e.g., to 4-bit precision) to further reduce memory usage and computational cost~\cite{dettmers2023qloraefficientfinetuningquantized}. QLoRA has proven particularly effective for fine-tuning Gemma-based models~\cite{gemma_qlora_2025}
      \item \textbf{Adapter Layers:} These involve inserting small, trainable neural network modules (adapters) between the existing layers of the pre-trained model. During fine-tuning, only the adapter parameters are updated, while the original model parameters remain frozen~\cite{hu2023llmadaptersadapterfamilyparameterefficient}.
      \item \textbf{Prefix Tuning:} This method adds a small set of trainable prefix tokens to the input sequence. The model learns to modulate its behavior based on these learned prefixes, without changing the core LLM parameters~\cite{li-liang-2021-prefix}.
    \end{itemize}
\end{itemize}

Instruction Fine-Tuning (IFT) is a specific type of fine-tuning that trains LLMs on datasets composed of (instruction, input, output) triplets. This helps the model learn to follow instructions better and perform specific tasks as directed. for QA tasks, this typically involves fine-tuning on (questions, context, answer) examples. In the context of RAG, IFT can be applied to:

\begin{itemize}
    \item Fine-tune the generator LLM to improve its ability to generate accurate and relevant answers based on the retrieved context or adhere to specific formatting requirements.
    \item Jointly train the retriever and generator components to improve their alignment and overall RAG performance.
    \item \textbf{RuleRAG:} is an example where symbolic rules are introduced as demonstration for in-context learning or as part of supervised fine-tuning data to explicitly guide both the retriever (to fetch logically related documents) and the generator (to produce answers that follow the rules)~\cite{chen2025ruleragruleguidedretrievalaugmentedgeneration}.
    \item \textbf{Join QA and Question Generation(QG):} Some research explores fine-tuning an LLM to perform both QA and QG tasks, allowing the model to generate questions based on the retrieved context and then answer them. This can enhance the model's understanding of the context and improve its ability to generate relevant answers~\cite{xu2025simragselfimprovingretrievalaugmentedgeneration}.
\end{itemize}

A crucial consideration in fine-tuning is the minimum dataset size required to achieve effective results. Recent studies, such as the LIMA paper~\cite{NEURIPS2023_ac662d74}, as cited in~\cite{ratnakar2025qapairsassessingparameterefficient}, suggest that even small but high-quality datasets (around 1000 samples) can yield significant improvements in performance when using PEFT techniques.
This is because most of the foundational knowledge is already acquired duirng the extensive pre-training phase. This finding makes the creation of suitable fine-tuning dataset for Indonesian consular QA an achieveable goal within the scope of this research.

\textbf{Compared Studies} often explore the trade-offs between fine-tuning alone, RAG alone, or a combination, or a combination of both (FT + RAG)~\cite{balaguer2024ragvsfinetuningpipelines,alghisi2024finetuneragevaluatingdifferent,ovadia2024finetuningretrievalcomparingknowledge}:
\begin{itemize}
    \item \textbf{RAG alone} excels in tasks requiring access to external, up-to-date, or proprietary knowledge. It is generally less prone to hallucination if the retriever is accurate and allows for easier knowledge updates by modifying the vector database rather retraining the LLM.
    \item \textbf{Fine-tuning alone} is effective for teaching the LLM new skils, adapting its style or tone, or instilling deep understanding of specific domain language and nuances. It can lead to higher accuracy on specialized task where the requirend knowledge can be embedded into the model's parameters. However, it risks catastrophic forgetting and the model's knowledge remains static based on its training data cu-off.
    \item \textbf{FT + RAG} is often considered the most effective approach, combining the strengths of boths. Fine-tuning can make the LLM a better reasoner or synthesizer over the contextual information provided by the RAG system. For example, the generator LLM with RAG pipeline can be fine-tuned to better handle the structure of retrieved documents, to follow specific instructions fir answer generation based on the domain's requirements, or to improve its faithfulness to the provided sources.
\end{itemize}

For fine-tuning Gemma-based models like SahabatAI, the QLoRA method is particularly promising. Libraries like Hugging Face Transformers, along with the PEFT and TRL (Transformer Reinforcement Learning) libraries (specifically the SFTTrainer), provide robust tools for implementing QLoRA fine-tuning for Gemma models~\cite{gemma_qlora_2025}. Keras, with backends like JAX, TensorFlow, or PyTorch, also offers support for LoRA tuning of Gemma models~\cite{gemma_lora_2025}.

The selection of QLoRA as the PEFT strategy for SahabatAI in this thesis is well supported by its efficiency with Gemma-based models. Its ability to handle the model's large parameter size, and its effectiveness in improving performance on domain-specific tasks with smaller datasets (around 1000 samples) further strengthens its suitability for this research. 

\section{Evaluation of RAG and QA Systems}

Evaluating the performance of complex AI systems like RAG-based question answering chatbots requires a multi-faceted approach, encompassing metrics that assess individual components as well as the end-to-end system behavior.

\textbf{Component-wise vs. End-to-End Evaluation:} RAG systems consist of distinct components, primarily the retriever and the generator (LLM). Evaluation can focus on the efficacy of each component in isolation (e.g., how well the retriever fetches relevant documents) or on the overall performance of the integrated system inanswering questions.

\textbf{Metrics for Retrieval Quality:} These metrics assess how effectively the retriever identifies and ranks relevant documents from the knowledge base in response to a query~\cite{manning2008ir, burges2005learning, pinecone2023evaluation}:
\begin{itemize}
    \item \textbf{Recall@k:} Measures the proportion of relevant documents retrieved within the top-k results. A higher recall indicates better retrieval performance.
    \item \textbf{Mean Reciprocal Rank (MRR):} Evaluates the average rank at which the first relevant document appears in the retrieved list. A lower rank indicates better retrieval performance.
    \item \textbf{Precision@k:} Assesses the proportion of relevant documents among the top-k retrieved documents. It provides insight into the quality of the retrieved set.
    \item \textbf{Normalized Discounted Cumulative Gain (NDCG):} This metric considers both the relevance and position of retrieved documents, rewarding relevant documents that appear earlier in the ranked list~\cite{evidently2023ndcg}.
\end{itemize}

\section{Metrics for Evaluating Large Language Model Output Quality}

The evaluation of answers generated by Large Language Models (LLMs), particularly in the context of queries and retrieved information, relies on a diverse set of metrics. These metrics assess various aspects of the generated text, from its lexical similarity to human-written answers to its factual accuracy and overall readability.

\begin{itemize}
  
  \item \textbf{Lexical Similarity (Reference-based)}

  These metrics quantify the quality of a generated answer by comparing it to one or more reference answers, typically authored by humans.

  \begin{itemize}
      \item \textbf{BLEU (Bilingual Evaluation Understudy):} Originating from machine translation, BLEU evaluates the concordance between machine-generated text and high-quality reference translations by measuring the overlap of n-grams (contiguous sequences of n items, typically words). It focuses on precision, assessing how much of the generated text's n-grams appear in the references, and includes a brevity penalty to discourage overly short outputs \cite{Papineni2002BLEU}. While widely used, its direct applicability to the nuanced quality of LLM generation, which often prioritizes coherence and semantic accuracy over exact n-gram matches, is a subject of ongoing discussion \cite{PremAI2024LLMEvalPart1}.

      \item \textbf{ROUGE (Recall-Oriented Understudy for Gisting Evaluation):} Frequently employed in summarization and question-answering evaluations, ROUGE assesses the quality of a generated text by comparing it to reference texts, focusing on recall. It counts the number of n-grams or subsequences from the reference text that are captured in the generated output \cite{Lin2004ROUGE}. Common variants include ROUGE-N, which considers n-gram overlap, and ROUGE-L, which measures the longest common subsequence to capture sentence-level structural similarity \cite{GalileoAI2025ROUGEMetric}.

      \item \textbf{METEOR (Metric for Evaluation of Translation with Explicit ORdering):} METEOR enhances upon simpler n-gram-based metrics by incorporating synonymy (matching words with similar meanings) and stemming (matching words to their root form). It performs an alignment between the generated and reference texts, calculating precision, recall, and a weighted F-score, while also applying a fragmentation penalty to account for word order and fluency \cite{Banerjee2005METEOR, AnalyticsVidhya2025Meteor}.
  \end{itemize}

  \item \textbf{Semantic Similarity}

  This category of metrics moves beyond lexical overlap to assess whether the meaning of the generated answer is similar to that of a reference answer or aligns with the expected meaning. This can be achieved by comparing the embeddings (vector representations) of the generated and reference answers or by employing another sophisticated model to judge the semantic equivalence between the texts \cite{Uppal2024SemanticSimilarity}. Such metrics are crucial as they can identify answers that are lexically different but semantically identical to the reference, which n-gram-based metrics might penalize.

  \item \textbf{Factuality \& Faithfulness (Groundedness)}

  Particularly critical for Retrieval Augmented Generation (RAG) systems, these metrics evaluate if the generated answer accurately reflects the information present in the provided source documents (the retrieved context). Faithfulness specifically ensures that the answer is exclusively supported by the given context and does not introduce external, unverified information or "hallucinations" \cite{Thulke2025FaithfulRAG}. Ensuring faithfulness is paramount for building trust and reliability in LLM-powered systems that rely on provided evidence.

  \item \textbf{Answer Relevance}

  Answer relevance metrics gauge whether the generated output directly and pertinently addresses the user's query. An answer might be fluent and factually correct based on some context, but if it does not answer the specific question asked, it is of little use. LLMs can be prompted or fine-tuned to assess relevance, often by comparing the generated answer against the user's question and sometimes the retrieved context \cite{Upadhyay2024RelevanceComparison, EvidentlyAI2025LLMasJudge}.

  \item \textbf{Coherence \& Fluency}

  These metrics assess the linguistic quality of the generated answer. Coherence refers to the logical connectedness and consistency of the ideas within the response, ensuring that it forms a sensible whole. Fluency pertains to the grammatical correctness, naturalness, and ease of understanding of the text \cite{Li2023LLMEval, Preprints2025LLMEvalReview}. A high-quality response should be easily readable and flow logically from one point to the next.

  \item \textbf{Completeness}

  Completeness evaluates if the generated answer sufficiently covers all important aspects of the user's question, given the provided context. While an answer might be relevant and factually accurate, it could still be incomplete if it omits crucial information necessary to fully address the query \cite{GalileoAI2024MetricsFirst}. This metric often works in tandem with context adherence to ensure comprehensive yet grounded responses.

  \item \textbf{Perplexity}

  Perplexity is an intrinsic evaluation metric that measures how well a language model predicts a given sample of text. A lower perplexity score generally indicates that the model is less "surprised" by the text, meaning it has a better ability to predict the sequence of tokens \cite{Comet2024Perplexity}. While historically significant for evaluating language model performance, its direct correlation with the quality of generated answers in downstream tasks can be limited, as it primarily reflects the model's confidence rather than factual correctness or task-specific utility \cite{AnalyticsVidhya2025Perplexity}.

  \item \textbf{LLM-as-a-Judge}

  This emerging evaluation paradigm utilizes a powerful, often proprietary, LLM (such as GPT-4) to assess the outputs of another LLM. The "judge" LLM is provided with the generated answer, the original prompt, and often the reference context, along with specific criteria (e.g., helpfulness, correctness, coherence, harmlessness) defined in a carefully crafted prompt \cite{Zheng2023JudgingLLM, ConfidentAI2025LLMJudgeExplained}. This approach offers a scalable alternative or complement to human evaluation, leveraging the advanced reasoning capabilities of state-of-the-art LLMs to score or rank generated text based on nuanced instructions \cite{ResearchGate2025SurveyLLMasJudge}.


  \item \textbf{End-to-End QA Performance}
    These metrics assess the overall ability of the system to answer questions correctly:

    \begin{itemize}
        \item \textbf{Exact Match (EM):} This metric measures the percentage of predictions that match one of the ground truth answers exactly, character for character. It is a stringent metric often used in extractive question answering tasks where the answer is typically a specific span of text from the context, or for questions requiring very short, precise answers \cite{Rajpurkar2016}.

        \item \textbf{F1-score:} In the context of question answering, particularly for evaluating answers that are spans of text, the F1-score is a commonly used metric that considers the token-level overlap between the predicted answer and the ground truth answer. It is calculated as the harmonic mean of precision (the proportion of tokens in the prediction that are in the ground truth) and recall (the proportion of tokens in the ground truth that are in the prediction), providing a more nuanced measure than exact match when partial correctness is relevant \cite{Rajpurkar2016}.

        \item \textbf{Human Evaluation:} Human evaluation is widely regarded as the most reliable method for assessing the multifaceted quality of answers generated by QA systems, especially for complex or generative tasks. Assessors typically rate answers based on a variety of criteria, which can include accuracy (factual correctness), completeness (coverage of the question's aspects), helpfulness (utility to the user), fluency (linguistic quality), and overall user satisfaction. Despite its high cost in terms of time and resources, human evaluation is often considered the gold standard because it can capture nuances of language understanding and relevance that automated metrics might miss \cite{Chang2023Survey, Belz2005HumanEval}.
    \end{itemize}

  \item \textbf{User-Centric Evaluation}
  These metrics focus on the user's experience and their ability to achieve their goals effectively and satisfactorily when interacting with the system:

  \begin{itemize}
      \item \textbf{Task Success Rate:} This metric quantifies the effectiveness of a system by measuring the percentage of users who successfully complete their intended tasks or find the specific information they are seeking. It is a fundamental measure of usability, indicating whether the system enables users to achieve their objectives \cite{Sauro2016}. In the context of chatbots or conversational AI, this would track if users successfully resolved their queries or completed transactions.

      \item \textbf{User Satisfaction (CSAT):} User satisfaction provides a measure of the user's subjective response to interacting with the system. It is commonly assessed through post-interaction surveys where users rate their contentment with the system, often using scales like Likert scales or single-question CSAT scores (e.g., "How satisfied were you with your interaction today?") \cite{Sauro2016, Følstad2018}. High user satisfaction is often correlated with continued use and positive perception of the system.

      \item \textbf{Other User Engagement Metrics:} Beyond immediate task success and satisfaction, a range of other metrics, often employed in monitoring production systems, can offer valuable insights into the overall utility, adoption, and engagement of a system like a chatbot. These include tracking the \textbf{Total Users} (overall reach), \textbf{Active Users} (e.g., daily, weekly, or monthly active users, indicating regular usage), and \textbf{Engaged Users} (users performing a certain number of interactions or spending a significant amount of time with the system). While not direct measures of answer quality per interaction, these metrics reflect the system's ability to retain users and its perceived value over time \cite{Williams2020, Zamora2021}.
  \end{itemize}

  \item \textbf{Benchmarking Datasets and Frameworks}
  The evaluation of question answering systems relies on a variety of benchmark datasets. Widely recognized standard datasets for general question answering include SQuAD \cite{Rajpurkar2016SQuAD}, Natural Questions \cite{Kwiatkowski2019NaturalQuestions}, TriviaQA \cite{Joshi2017TriviaQA}, and MS MARCO \cite{Campos2016MSMARCO}. However, for domain-specific question answering, the development of custom datasets is often necessary to adequately assess performance in specialized areas. An example of such a specialized dataset is ChemLit-QA, which is tailored for the chemistry domain \cite{Corrales2024ChemLitQA}.

  Particularly for Retrieval Augmented Generation (RAG) systems, the need for high-quality, domain-specific evaluation datasets is a critical area of focus to ensure robust and reliable performance assessment \cite{Es2023RAGAS, SaadFalcon2023ARES}. To address the complexities of evaluating RAG pipelines, several specialized frameworks have been developed. These frameworks aim to streamline the evaluation process by providing tools and methodologies for assessing various aspects of RAG systems. Notable examples include:
  \begin{itemize}
      \item RAGAS \cite{Es2023RAGAS}
      \item ARES \cite{SaadFalcon2023ARES}
      \item DeepEval \cite{Li2024DeepEval}
      \item UpTrain \cite{UpTrainAI}
      \item Tonic Validate \cite{TonicValidate}
  \end{itemize}

  \item \textbf{Challenges in Evaluating Low-Resource Language LLMs}
  The evaluation of Large Language Models (LLMs) for low-resource languages presents distinct challenges. These primarily stem from the scarcity of appropriate benchmarks and the critical need for evaluation materials and criteria that are not only linguistically accurate but also culturally relevant to the specific language communities \cite{Chang2023Survey}. Addressing these challenges is crucial for ensuring that LLMs are equitable and perform effectively across a diverse range of languages.

\item \textbf{Evaluation Strategy for an AI-Powered Consular Information System}
  Given the high-stakes nature of consular information, where accuracy, reliability, and clarity are paramount, a multi-faceted evaluation strategy is indispensable for the proposed AI-powered system. While automated lexical similarity metrics like ROUGE \cite{Lin2004ROUGE} and BLEU \cite{Papineni2002BLEU} can offer rapid feedback during the iterative development process, they are, by themselves, insufficient to capture the full spectrum of quality required for such a sensitive application. Therefore, a greater emphasis must be placed on metrics that assess the nuanced aspects of the generated content. Key among these are faithfulness, ensuring the information strictly adheres to consular regulations (groundedness) \cite{Thulke2025FaithfulRAG}, alongside answer relevance to the user's query \cite{Upadhyay2024RelevanceComparison}, and the overall coherence of the provided information \cite{Li2023LLMEval}. An LLM-as-a-Judge approach \cite{Zheng2023JudgingLLM} could prove valuable for assessing these more qualitative aspects at scale. Ultimately, however, human evaluation will be an essential component of the validation process. Even if conducted on a smaller, representative subset of queries, human assessment is critical for confirming the practical helpfulness, accuracy, and overall trustworthiness of the AI-powered consular chatbot before deployment \cite{Chang2023Survey}.


\end{itemize}

\section{AI in Government and Consular Services}

The adoption of AI in government and public services has gained momentum in recent years, with numerous countries implementing AI solutions to enhance efficiency, accessibility, and citizen engagement. 
Global Case Studies provide valuable insights into the potential and challenges of AI:
\begin{itemize}
    \item \textbf{Singapore:} Since the launch of AI-powered chatbots like Ask Jamie, HealthBuddy, and the CPF Chatbot in the mid-2010s, Singapore has progressively enhanced its use of AI in public services. These early systems utilized Natural Language Processing (NLP) to handle citizen inquiries across multiple languages, significantly reducing call center workloads and improving response efficiency~\cite{govtech2016askjamie}. Building on this foundation, Singapore has developed more advanced AI tools such as Pair, an AI assistant designed to boost productivity among public officers, and VICA, a conversational AI platform serving over 60 government agencies and handling hundreds of thousands of queries monthly~\cite{govtech2025pair}. Additionally, specialized AI applications like the OneService Chatbot, which routes municipal feedback with about 80\% accuracy, and PEACH, an AI chatbot aiding perioperative clinical decisions with over 96\% accuracy, demonstrate Singapore's commitment to integrating AI for both operational efficiency and domain-specific expertise~\cite{chong2024peach}. Together, these developments illustrate Singapore's trajectory from basic NLP chatbots to sophisticated, large-scale AI deployments driving a smarter, more responsive public sector. \cite{govsgvica2025}.
    \item \textbf{Japan:} Since 2020, Japan has made significant strides in integrating large language models (LLMs) into its healthcare sector. In April 2025, researchers at the National Institute of Informatics developed a generative AI system that successfully passed the national medical licensing examination. Trained on over 700,000 licensed medical papers, 16 million documents from reputable medical websites, and medical textbooks, this LLM is designed to assist clinicians by suggesting potential diagnoses based on patient interviews. The government plans to deploy this model in medical institutions to support clinical decision-making and improve operational efficiency~\cite{nii2025ai,crds2024rwd}
    \item \textbf{European Union:} The \textbf{iBorderCtrl} project,  was an EU-funded initiative under the Horizon 2020 program, running from September 2016 to August 2019. The project aimed to enhance the efficiency and security of land border crossings for third-country nationals entering the Schengen Area. The system introduced a two-stage process: pre-travel registration and on-site border control. However, the iBorderCtrl project faced significant criticism regarding its ethical implications and scientific validity. Concerns were raised about the reliability of AI-based lie detection, potential biases in facial recognition technology, and the risk of infringing on individuals' rights to privacy and fair treatment. A study published highlighted statistical and methodological flaws in the system's design, questioning the effectiveness of using "biomarkers of deceit" for mass screening purposes~\cite{sánchezmonedero2020politicsdeceptivebordersbiomarkers}. 
    \textbf{AskThePublic}, an AI-powered chatbot integrated into the "Have Your Say" platform. This tool allows citizens to interactively explore and contribute to EU policy consultations. By leveraging large language models, AskThePublic provides structured responses and improved language capabilities, thereby increasing public engagement in the policymaking process. A study published in April 2025 highlights its effectiveness in fostering inclusive decision-making and enhancing democratic participation~\cite{sprenkamp2025effectiveeueparticipationdevelopment}.
    The 2022 report AI Watch~\cite{doi/10.2760/39336} provides a comprehensive analysis of AI adoption across EU public administrations. Conducted by the European Commission's Joint Research Centre, the study maps AI use in public services through three main pillars: an analysis of national AI strategies, an inventory of 686 AI use cases across EU Member States, and in-depth case studies. Findings indicate that while a third of AI initiatives are operational, many remain in pilot or development stages. AI applications predominantly support public service engagement, policy analysis, and internal management. The report emphasizes that successful AI adoption is driven by national governments' capacity and highlights the significant role of regional and local administrations. It also underscores challenges such as data governance, technical capacity, and ethical considerations, advocating for a responsible and inclusive approach to AI integration in the public sector.
    \item \textbf{United States:} The U.S. federal government has significantly expanded its use of artificial intelligence (AI) across public services, aiming to enhance efficiency, decision-making, and service delivery~\cite{ncsl2024ai}. As of 2024, agencies reported 1,757 AI use cases, more than doubling the previous year's count. Notable implementations include the U.S. Patent and Trademark Office employing AI to streamline patent classification, the State Department utilizing AI for document translation and policy summarization, and the Transportation Security Administration integrating AI to expedite airport screenings~\cite{brookings2024ai} . To ensure responsible AI adoption, the Office of Management and Budget (OMB) issued policy M-24-10, mandating agencies to designate Chief AI Officers, implement risk management practices, and enhance transparency in AI applications\cite{OMB-M-24-10}.
    
    Despite these advancements, challenges persist. The Government Accountability Office (GAO) identified that many agencies have yet to fully comply with federal AI requirements, such as maintaining comprehensive AI inventories and updating workforce classifications to include AI competencies. The GAO recommended that 19 agencies take corrective actions to align with established AI policies and guidance~\cite{gao2024ai} . Additionally, the National Conference of State Legislatures (NCSL) reported that over 30 states have issued AI-related guidance, focusing on ethical frameworks, oversight, and pilot projects to guide AI integration at the state level~\cite{ncsl2024ai}

    \item \textbf{Indonesia:} The Indonesian government is actively pursuing the integration of Artificial Intelligence (AI) into various public sectors, guided by its National Strategy for Artificial Intelligence 2020-2045 \cite{mfat_indonesia_ai_strategy_2023, investjakarta_ai_global_trends_2025}. This strategy prioritizes AI development in key areas including healthcare, bureaucratic reform, education and research, food security, and mobility and smart cities \cite{mfat_indonesia_ai_strategy_2023}. For instance, the government plans to launch an AI application in August 2025 to bolster food security and social protection programs \cite{techinasia_indonesia_ai_2025}, alongside developing AI-based health screening services to improve healthcare accessibility \cite{govinsider_navigating_ai_2025}. The Ministry of Communication and Digital Affairs (Kominfo) is spearheading efforts in AI governance, issuing ethical guidelines for AI use and working on more comprehensive regulations to ensure responsible AI adoption \cite{opengovasia_indonesia_ai_strategy_2025, govinsider_navigating_ai_2025}. Furthermore, the National Research and Innovation Agency (BRIN) has established an AI and Cybersecurity Research Centre and is involved in drafting AI-related presidential regulations \cite{opengovasia_indonesia_ai_strategy_2025}. 
    
    Several concrete examples of AI implementation are emerging. The Ministry of Finance has launched "Coretax," an AI-supported system aimed at automating and integrating core tax administration processes \cite{dipstrategy_fenomena_penerapan_ai}. The MoFA launch SARI Chatbot.

    Despite these advancements, Indonesia faces challenges such as bridging the digital talent gap, ensuring adequate digital infrastructure across its vast archipelago, and navigating budgetary constraints for technological adoption \cite{govinsider_navigating_ai_2025, mfat_indonesia_ai_strategy_2023}. The government acknowledges these hurdles and emphasizes a socio-technical approach, focusing on developing digital talent, fostering public-private partnerships, and creating a robust regulatory framework to maximize AI's benefits while mitigating potential risks \cite{govinsider_navigating_ai_2025, opengovasia_indonesia_ai_strategy_2025}.

    UNESCO and Indonesia's Ministry of Communications and Informatics (KOMINFO) completed an AI Readiness Assessment for Indonesia, making it the first country in Southeast Asia to undergo such an evaluation. This comprehensive assessment engaged over 500 participants and examined Indonesia's AI landscape across various dimensions, including legal, socio-cultural, economic, scientific, educational, and technical aspects. The resulting report highlights critical areas for Indonesia's AI readiness, such as economic and socio-cultural impacts, associated risks, and research funding, while also proposing key policy recommendations like the establishment of a National Agency for Artificial Intelligence and ensuring equitable access to AI education and resources~\cite{unescoina2024ai, unescoina2024aireport}.
    
\end{itemize}

\newpage

% Chapter 3: Methodology
\chapter{Methodology}
\section{Research Design}
Outline the research design, including the type of study (e.g., experimental, observational) and the methods used to collect and analyze data.
\section{Data Collection}
Describe how you collected data for your research, including any surveys, interviews, or experiments conducted.
\begin{figure}
    \centering
    \includegraphics[width=0.8\textwidth]{images/1/apa-ini.png}
    \caption{The typical RAG pipeline.}
    \label{fig:example2}
\end{figure}

\lipsum[1-2] % Dummy text for illustration

\begin{table}[htbp]
  \centering
  \caption{Example of a table caption.}
  \label{tab:example1}
  \begin{tabular}{|l|c|r|}
    \hline
    Column 1 & Column 2 & Column 3 \\
    \hline
    Data A   & 123      & X \\
    Data B   & 456      & Y \\
    Data C   & 789      & Z \\
    \hline
  \end{tabular}
\end{table}

\begin{table}[htbp]
  \centering
  \caption{Example of a table caption.}
  \label{tab:example2}
  \begin{tabular}{lcr}
    \toprule
    Column 1 & Column 2 & Column 3 \\
    \midrule
    Data A   & 123      & X \\
    Data B   & 456      & Y \\
    Data C   & 789      & Z \\
    \bottomrule
  \end{tabular}
\end{table}

\section{Data Analysis}
Explain the methods used to analyze the data, including any statistical tests or software used.
\section{Implementation}
Describe how you implemented the proposed system, including any algorithms or models used.
\section{Evaluation}
Explain how you evaluated the performance of your system, including any metrics used to measure accuracy, precision, recall, etc.
\section{Ethical Considerations}
Discuss any ethical considerations related to your research, including data privacy, consent, and potential biases in the model.

\newpage
% Chapter 4: Results and Discussion
\chapter{Results and Discussion}
Present the results of your research, along with discussions on the implications and findings.

\chapter{Conclusion}
Summarize the key findings of your research and suggest future work or improvements.

\newpage
% References
\sloppy
\printbibliography
\newpage

% Appendices
\appendix
\chapter{Additional Information}
Include any additional data, charts, or detailed explanations that are important for understanding your research.

\end{document}